
\documentclass[11pt,a4paper]{article}

% Packages
\usepackage[utf8]{inputenc}
\usepackage{amsmath, amssymb}
\usepackage{graphicx}
\usepackage{hyperref}
\usepackage{geometry}
\geometry{margin=1in}

% Title and author
\title{Deep Learning for Maximum Cut}
\author{Victor Pekkari \\ \small{University of California, San Diego} \\ \small{\texttt{epekkari@ucsd.edu}}}
\date{\today}

\begin{document}

\maketitle

\begin{abstract}
% Background/Context:
% What is the general area or problem being addressed?

% Objective/Goal:
% What is the specific problem or question your paper tackles?

% Methods/Approach:
% What approach, method, or model did you use?

% Results:
% What are the main findings or outcomes?

% Conclusion/Significance:
% What is the impact or importance of your results?
The purpose of this study is to explore the application of deep learning techniques to the Maximum Cut problem, a fundamental problem in graph theory with significant implications in various fields such as computer science, physics, and operations research. The objective is to develop a neural network model that can effectively approximate solutions to the Maximum Cut problem, which is known to be NP-hard.
We explore three different neural network architectures for solving maximum cut and prepare them to heuristic methods like semidefinite programming. 
Our results indicate that different methods for solving Max-Cut are optimal for different types of networks, highlighting the importance of selecting the appropriate approach based on the structure of the input graph.
\end{abstract}

\section{Introduction}
Introduce the problem, motivation, and background. Explain why this work is important.

\section{Background}
Discuss previous work and how your work is different or builds upon it.

% Continue with more sections as needed

\end{document}